\section{Ridge Regression} \label{sc:ridge}

\subsection{Background}
Similarly with Lasso, Ridge regression is an extension of OLS with regularization term. The difference between Lasso and Ridge is the L-norm regularization. Instead of L1 norm, Ridge uses the L2 norm for its regularization term.
\begin{equation}
\label{eq:l2_regularization}
\lambda \sum_{i=0}^{N} (w_i)^2
\end{equation}
The L2 regularization term is defined as the summation of the square of the weight for each features, multiplied by the penalty weight of $\lambda$ (default value 0.1). The L2 regularization term is included to the error term in OLS regression.

\subsection{Implementation}
The implementation of Ridge regression model is done using Scikit Learn Library in Python, along with supporting libraries such as Pandas and Numpy.

\subsection{Preprocessing}
The data used in this experiment is derived from preprocessed data in chapter \ref{ch:preprocessing} of this report, where \textit{StateHoliday, Storetype, Assortment} are all converted into one hot encoding. \\ \\
Additionally, \textit{CompetitionDistance} feature is normalized using min-max normalization method. \textit{Customers} data is dropped from training set as it does not exist in the testing data. Furthermore, \textit{Date} feature is also dropped from training data as we want to ensure that all features are in the form of integer, boolean, or float.

\subsection{Model Training \& Testing}
Sales forecast is only done for open stores (i.e. \textit{Open} equals to true). For closed stores, the predicted sales value will be set to 0.

\subsubsection{Normalized Train Test}
In this method, \textit{Sales} data is normalized using global mean and standard deviation. The resulting predictions are denormalized using the same mean and standard deviation.
By using a total of 20 features for training and testing, the result from Kaggle is summarized in table \ref{tab:ridge_norm_result}.
\begin{table}[h]
	\centering
	\caption{Normalized Ridge Result}
	\label{tab:ridge_norm_result}
	\begin{tabular}{|m{100pt}|m{50pt}|}
		\hline
		Private score & 0.21149 \\ \hline
		Public score  & 0.20215 \\ \hline
	\end{tabular}
\end{table}
\\
Following the experiment, further normalization is done by normalizing \textit{Sales} with regards to the mean and standard deviation of each store.
\begin{table}[h]
	\centering
	\caption{Store Normalized Ridge Result}
	\label{tab:ridge_norm_result_2}
	\begin{tabular}{|m{100pt}|m{50pt}|}
		\hline
		Private score & 0.20079 \\ \hline
		Public score  & 0.18565 \\ \hline
	\end{tabular}
\end{table}

\subsubsection{Train Test per Store}
Using the store-normalized \textit{Sales} data, we defined a specific Ridge model for each \textit{StoreId}. The result of this method can be observed in table \ref{tab:ridge_per_store_result}.
\begin{table}[h]
	\centering
	\caption{Ridge Result per Store}
	\label{tab:ridge_per_store_result}
	\begin{tabular}{|m{100pt}|m{50pt}|}
		\hline
		Private score & 0.17369 \\ \hline
		Public score  & 0.15602 \\ \hline
	\end{tabular}
\end{table}

\subsubsection{K-Fold Cross Validation}
To further improve the performance of Ridge regression, K-fold cross validation is used in the training phase. \\ \\
Using a 3-fold cross validation method, the model with best validation score is selected to predict the \textit{Sales} for each \textit{StoreId}. The result can be observed in table \ref{tab:ridge_3_fold}.
\begin{table}[H]
	\centering
	\caption{3-Fold Validation Ridge Result}
	\label{tab:ridge_3_fold}
	\begin{tabular}{|m{100pt}|m{50pt}|}
		\hline
		Private score & 0.17637 \\ \hline
		Public score  & 0.15988 \\ \hline
	\end{tabular}
\end{table}

\noindent
By increasing the number of fold to 5, the improvement of the prediction result can be seen in table \ref{tab:ridge_5_fold}.
\begin{table}[H]
	\centering
	\caption{5-Fold Validation Ridge Result}
	\label{tab:ridge_5_fold}
	\begin{tabular}{|m{100pt}|m{50pt}|}
		\hline
		Private score & 0.17318 \\ \hline
		Public score  & 0.15527 \\ \hline
	\end{tabular}
\end{table}

\noindent
Using 10 fold validation, the resulting score is shown in table \ref{tab:ridge_10_fold}.
\begin{table}[H]
	\centering
	\caption{10-Fold Validation Ridge Result}
	\label{tab:ridge_10_fold}
	\begin{tabular}{|m{100pt}|m{50pt}|}
		\hline
		Private score & 0.17387 \\ \hline
		Public score  & 0.15628 \\ \hline
	\end{tabular}
\end{table}

\subsubsection{Extra Feature}
This method performs a 5-fold validation on the previous dataset with additional features. \\ \\
Using a one hot encoding for \textit{DayOfWeek}, a total of 26 features are used for the predictive model.
\begin{table}[h]
	\centering
	\caption{Ridge Result with 26 Features}
	\label{tab:ridge_extra_feature_1}
	\begin{tabular}{|m{100pt}|m{50pt}|}
		\hline
		Private score & 0.16308 \\ \hline
		Public score  & 0.14370 \\ \hline
	\end{tabular}
\end{table}
\\
The historical data for \textit{Customers} and \textit{Sales} is used to derive additional features such as \textit{Customers\_Max, Customers\_Min, Customers\_Avg, Sales\_Max, Sales\_Min, Sales\_Avg}, where the additional features are relative to \textit{StoreId}. The result of using 32 features can be observed in table \ref{tab:ridge_extra_feature_2}.
\begin{table}[h]
	\centering
	\caption{Ridge Result with 32 Features}
	\label{tab:ridge_extra_feature_2}
	\begin{tabular}{|m{100pt}|m{50pt}|}
		\hline
		Private score & 0.16309 \\ \hline
		Public score  & 0.14367 \\ \hline
	\end{tabular}
\end{table}

\subsubsection{Variations of Penalty Weight ($\lambda$)}
The experiments above are repeated using different $\lambda$ values, such as $10^{-15}$, $10^{-10}$, $10^{-8}$, $10^{-4}$, $10^{-3}$, 0.01, 1, 5, 10, 20.
However, all variation of $\lambda$ produced similar or worse result compared with the default $\lambda$ value of 0.1.
