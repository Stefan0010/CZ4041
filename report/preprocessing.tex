\chapter{Preprocessing} \label{ch:preprocessing}
The problem contains three files, \textit{train.csv}, \textit{test.csv}, and \textit{store.csv}. The file \textit{train.csv} contains historical data including sales for 1115 stores everyday from 1 Jan 2013 to 31 July 2015. While \textit{store.csv} contains supplemental information about each store. Then, \textit{test.csv} contains historical data excluding sales and number of customers everyday from 1 Aug 2015 to 17 Sept 2015. \\ \\
Each row in \textit{train.csv} contains store ID, day of week, date, number of customers, sales, whether the store is open, whether the store is doing a promo, state holiday, and whether that day is a school holiday. Columns in \textit{test.csv} are almost identical to that of \textit{train.csv}, except that sales and number of customers are unknown in \textit{test.csv}. Each row in \textit{test.csv} contains a submission ID for the purpose of evaluation on prediction result. On the other hand, each row in \textit{store.csv} contains the details of a store, such as store type, assortment type, whether the store has competition, since when the competition exists, competition distance, whether the store is doing \textit{promo2}, and \textit{promo2} period. \\ \\
Initially, our preprocessing method merges \textit{train.csv} and \textit{store.csv} by store ID. The columns \textit{DayOfWeek} and \textit{StateHoliday} in \textit{train.csv}  are transformed into one-hot vector respectively. The columns \textit{StoreType} and \textit{Assortment} in \textit{store.csv} are transformed into one-hot vector respectively as well. The columns \textit{CompetitionOpenSinceMonth} and \textit{CompetitionOpenSinceYear} are substituted into a single column \textit{HasCompetition} which depends on its respective \textit{Date} column. The same thing also applies to the columns \textit{Promo2SinceWeek}, \textit{Promo2SinceYear}, and \textit{PromoInterval}, they are substituted into a single column \textit{IsDoingPromo2} which depends in its respective \textit{Date} column. \textit{CompetitionDistance} is set to the maximum value in the training set if a store does not have a competition in any given date. All store data are retained even when a store is closed on a particular date. \\ \\
After \textit{train.csv} and \textit{store.csv} has been merged, we proceed to merge \textit{test.csv} and \textit{store.csv} by store ID as well. The preprocessing method in this step is identical in the previous paragraph. The difference is that sales and the number of customers are unknown in the test dataset.